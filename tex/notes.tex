\documentclass[11pt]{article}

\usepackage[utf8]{inputenc}
\usepackage{graphicx}
\usepackage{array}
\usepackage{amsmath}
\usepackage{xcolor}
\pagecolor{white}
\usepackage{geometry}
\geometry{a4paper}


\begin{document}


\section{Zolotarev Notes}

The Zolotarev optimal rational function approximating the sign function $\mathcal{Z}(x)=1$ over the range $x\in[r_1,r_2]$ is expressed in factored form by
\begin{equation}
\mathcal{Z}(x)=mx\frac{\prod_{j=1}^{j=N_n}x^2-a_i}{\prod_{j=1}^{N_d}x^2-d_i} 
\end{equation}
and in partial fraction form by
\begin{equation}
\mathcal{Z}(x)=mx[f+\sum_{j=1}^{N_d}\frac{c_i}{x^2-d_i}]
\end{equation}

Parameters $r_1$,$r_2$, and $N$ are hard coded in to main.cpp in the function z.setZolo(r1,r2,N). The coefficients are output in zoloFactorCoeffs.dat and zolozoloPartFracCoeffs.dat respectively.

\subsection{Overlap Operator}

The overlap operator (with Wilson kernel) is then expressed directly as
\begin{equation}
D_{OL}=\frac{1+m}{2}+\frac{1-m}{2}mD_W( f+ \sum_j \frac{c_j} {D_W^\dagger D_W-d_j})
\end{equation}
where $D_W\equiv D_W(M)$ is the usual Wilson Dirac operator with parameter $M\in(0,2)$.

\subsection{Domain Wall}

The massless domain wall operator with Wilson kernel is illustrated by

\begin{equation}
D_{DW} = 
\begin{pmatrix}
 \omega_1 D^\parallel+I & (\omega_1 D^\parallel-I)P_- & 0 & 0 \\
 (\omega_2 D^\parallel-I)P_+ & \omega_2 D^\parallel+I & (\omega_2 D^\parallel-I)P_- & 0 \\
 0 & (\omega_3 D^\parallel-I)P_+ & \omega_3 D^\parallel+I & (\omega_3 D^\parallel-I)P_- \\
 0 & 0 & (\omega_4 D^\parallel-I)P_+ & \omega_4 D^\parallel+I 
\end{pmatrix}
\end{equation}

The coefficients $\omega_i$ are set to 1 to be equivalent to a hyperbolic tangent approximation of the sign function. To be equivalent to the Zolotarev approximation we set $\omega_i=1/u_u$ where $u_i$ are the roots of $\mathcal{Z}(x)-1$. These roots are output in the file zroots.dat. This indirect form is more efficient in the construction of sea fermions in the RHMC algorithm for instance.


\section{Hyperbolic Tangent}

  Partial fraction and factored coefficients are also output for the hyperbolic tangent approximation if required. This may be the preferable (and simpler) option when weakly coupled, with a small condition number, but is inefficient for stronger coupling with large condition number.
  


\end{document}

